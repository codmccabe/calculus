\documentclass{ximera}

%\usepackage{todonotes}
%\usepackage{mathtools} %% Required for wide table Curl and Greens
%\usepackage{cuted} %% Required for wide table Curl and Greens
\newcommand{\todo}{}

\usepackage{esint} % for \oiint
\ifxake%%https://math.meta.stackexchange.com/questions/9973/how-do-you-render-a-closed-surface-double-integral
\renewcommand{\oiint}{{\large\bigcirc}\kern-1.56em\iint}
\fi


\graphicspath{
  {./}
  {ximeraTutorial/}
  {basicPhilosophy/}
  {functionsOfSeveralVariables/}
  {normalVectors/}
  {lagrangeMultipliers/}
  {vectorFields/}
  {greensTheorem/}
  {shapeOfThingsToCome/}
  {dotProducts/}
  {partialDerivativesAndTheGradientVector/}
  {../productAndQuotientRules/exercises/}
  {../motionAndPathsInSpace/exercises/}
  {../normalVectors/exercisesParametricPlots/}
  {../continuityOfFunctionsOfSeveralVariables/exercises/}
  {../partialDerivativesAndTheGradientVector/exercises/}
  {../directionalDerivativeAndChainRule/exercises/}
  {../commonCoordinates/exercisesCylindricalCoordinates/}
  {../commonCoordinates/exercisesSphericalCoordinates/}
  {../greensTheorem/exercisesCurlAndLineIntegrals/}
  {../greensTheorem/exercisesDivergenceAndLineIntegrals/}
  {../shapeOfThingsToCome/exercisesDivergenceTheorem/}
  {../greensTheorem/}
  {../shapeOfThingsToCome/}
  {../separableDifferentialEquations/exercises/}
  {vectorFields/}
}

\newcommand{\mooculus}{\textsf{\textbf{MOOC}\textnormal{\textsf{ULUS}}}}

\usepackage{tkz-euclide}\usepackage{tikz}
\usepackage{tikz-cd}
\usetikzlibrary{arrows}
\tikzset{>=stealth,commutative diagrams/.cd,
  arrow style=tikz,diagrams={>=stealth}} %% cool arrow head
\tikzset{shorten <>/.style={ shorten >=#1, shorten <=#1 } } %% allows shorter vectors

\usetikzlibrary{backgrounds} %% for boxes around graphs
\usetikzlibrary{shapes,positioning}  %% Clouds and stars
\usetikzlibrary{matrix} %% for matrix
\usepgfplotslibrary{polar} %% for polar plots
\usepgfplotslibrary{fillbetween} %% to shade area between curves in TikZ
%\pgfplotsset{compat=1.14}
%\usetkzobj{all} obsolete

\usepackage[makeroom]{cancel} %% for strike outs
%\usepackage{mathtools} %% for pretty underbrace % Breaks Ximera
%\usepackage{multicol}
\usepackage{pgffor} %% required for integral for loops



%% http://tex.stackexchange.com/questions/66490/drawing-a-tikz-arc-specifying-the-center
%% Draws beach ball
\tikzset{pics/carc/.style args={#1:#2:#3}{code={\draw[pic actions] (#1:#3) arc(#1:#2:#3);}}}



\usepackage{array}
\setlength{\extrarowheight}{+.1cm}
\newdimen\digitwidth
\settowidth\digitwidth{9}
\def\divrule#1#2{
\noalign{\moveright#1\digitwidth
\vbox{\hrule width#2\digitwidth}}}





\newcommand{\RR}{\mathbb R}
\newcommand{\R}{\mathbb R}
\newcommand{\N}{\mathbb N}
\newcommand{\Z}{\mathbb Z}

\newcommand{\sagemath}{\textsf{SageMath}}


%\renewcommand{\d}{\,d\!}
\renewcommand{\d}{\mathop{}\!d}
\newcommand{\dd}[2][]{\frac{\d #1}{\d #2}}
\newcommand{\pp}[2][]{\frac{\partial #1}{\partial #2}}
\renewcommand{\l}{\ell}
\newcommand{\ddx}{\frac{d}{\d x}}

\newcommand{\zeroOverZero}{\ensuremath{\boldsymbol{\tfrac{0}{0}}}}
\newcommand{\inftyOverInfty}{\ensuremath{\boldsymbol{\tfrac{\infty}{\infty}}}}
\newcommand{\zeroOverInfty}{\ensuremath{\boldsymbol{\tfrac{0}{\infty}}}}
\newcommand{\zeroTimesInfty}{\ensuremath{\small\boldsymbol{0\cdot \infty}}}
\newcommand{\inftyMinusInfty}{\ensuremath{\small\boldsymbol{\infty - \infty}}}
\newcommand{\oneToInfty}{\ensuremath{\boldsymbol{1^\infty}}}
\newcommand{\zeroToZero}{\ensuremath{\boldsymbol{0^0}}}
\newcommand{\inftyToZero}{\ensuremath{\boldsymbol{\infty^0}}}



\newcommand{\numOverZero}{\ensuremath{\boldsymbol{\tfrac{\#}{0}}}}
\newcommand{\dfn}{\textbf}
%\newcommand{\unit}{\,\mathrm}
\newcommand{\unit}{\mathop{}\!\mathrm}
\newcommand{\eval}[1]{\bigg[ #1 \bigg]}
\newcommand{\seq}[1]{\left( #1 \right)}
\renewcommand{\epsilon}{\varepsilon}
\renewcommand{\phi}{\varphi}


\renewcommand{\iff}{\Leftrightarrow}

\DeclareMathOperator{\arccot}{arccot}
\DeclareMathOperator{\arcsec}{arcsec}
\DeclareMathOperator{\arccsc}{arccsc}
\DeclareMathOperator{\si}{Si}
\DeclareMathOperator{\scal}{scal}
\DeclareMathOperator{\sign}{sign}


%% \newcommand{\tightoverset}[2]{% for arrow vec
%%   \mathop{#2}\limits^{\vbox to -.5ex{\kern-0.75ex\hbox{$#1$}\vss}}}
\newcommand{\arrowvec}[1]{{\overset{\rightharpoonup}{#1}}}
%\renewcommand{\vec}[1]{\arrowvec{\mathbf{#1}}}
\renewcommand{\vec}[1]{{\overset{\boldsymbol{\rightharpoonup}}{\mathbf{#1}}}\hspace{0in}}

\newcommand{\point}[1]{\left(#1\right)} %this allows \vector{ to be changed to \vector{ with a quick find and replace
\newcommand{\pt}[1]{\mathbf{#1}} %this allows \vec{ to be changed to \vec{ with a quick find and replace
\newcommand{\Lim}[2]{\lim_{\point{#1} \to \point{#2}}} %Bart, I changed this to point since I want to use it.  It runs through both of the exercise and exerciseE files in limits section, which is why it was in each document to start with.

\DeclareMathOperator{\proj}{\mathbf{proj}}
\newcommand{\veci}{{\boldsymbol{\hat{\imath}}}}
\newcommand{\vecj}{{\boldsymbol{\hat{\jmath}}}}
\newcommand{\veck}{{\boldsymbol{\hat{k}}}}
\newcommand{\vecl}{\vec{\boldsymbol{\l}}}
\newcommand{\uvec}[1]{\mathbf{\hat{#1}}}
\newcommand{\utan}{\mathbf{\hat{t}}}
\newcommand{\unormal}{\mathbf{\hat{n}}}
\newcommand{\ubinormal}{\mathbf{\hat{b}}}

\newcommand{\dotp}{\bullet}
\newcommand{\cross}{\boldsymbol\times}
\newcommand{\grad}{\boldsymbol\nabla}
\newcommand{\divergence}{\grad\dotp}
\newcommand{\curl}{\grad\cross}
%\DeclareMathOperator{\divergence}{divergence}
%\DeclareMathOperator{\curl}[1]{\grad\cross #1}
\newcommand{\lto}{\mathop{\longrightarrow\,}\limits}

\renewcommand{\bar}{\overline}

\colorlet{textColor}{black}
\colorlet{background}{white}
\colorlet{penColor}{blue!50!black} % Color of a curve in a plot
\colorlet{penColor2}{red!50!black}% Color of a curve in a plot
\colorlet{penColor3}{red!50!blue} % Color of a curve in a plot
\colorlet{penColor4}{green!50!black} % Color of a curve in a plot
\colorlet{penColor5}{orange!80!black} % Color of a curve in a plot
\colorlet{penColor6}{yellow!70!black} % Color of a curve in a plot
\colorlet{fill1}{penColor!20} % Color of fill in a plot
\colorlet{fill2}{penColor2!20} % Color of fill in a plot
\colorlet{fillp}{fill1} % Color of positive area
\colorlet{filln}{penColor2!20} % Color of negative area
\colorlet{fill3}{penColor3!20} % Fill
\colorlet{fill4}{penColor4!20} % Fill
\colorlet{fill5}{penColor5!20} % Fill
\colorlet{gridColor}{gray!50} % Color of grid in a plot

\newcommand{\surfaceColor}{violet}
\newcommand{\surfaceColorTwo}{redyellow}
\newcommand{\sliceColor}{greenyellow}




\pgfmathdeclarefunction{gauss}{2}{% gives gaussian
  \pgfmathparse{1/(#2*sqrt(2*pi))*exp(-((x-#1)^2)/(2*#2^2))}%
}


%%%%%%%%%%%%%
%% Vectors
%%%%%%%%%%%%%

%% Simple horiz vectors
\renewcommand{\vector}[1]{\left\langle #1\right\rangle}


%% %% Complex Horiz Vectors with angle brackets
%% \makeatletter
%% \renewcommand{\vector}[2][ , ]{\left\langle%
%%   \def\nextitem{\def\nextitem{#1}}%
%%   \@for \el:=#2\do{\nextitem\el}\right\rangle%
%% }
%% \makeatother

%% %% Vertical Vectors
%% \def\vector#1{\begin{bmatrix}\vecListA#1,,\end{bmatrix}}
%% \def\vecListA#1,{\if,#1,\else #1\cr \expandafter \vecListA \fi}

%%%%%%%%%%%%%
%% End of vectors
%%%%%%%%%%%%%

%\newcommand{\fullwidth}{}
%\newcommand{\normalwidth}{}



%% makes a snazzy t-chart for evaluating functions
%\newenvironment{tchart}{\rowcolors{2}{}{background!90!textColor}\array}{\endarray}

%%This is to help with formatting on future title pages.
\newenvironment{sectionOutcomes}{}{}



%% Flowchart stuff
%\tikzstyle{startstop} = [rectangle, rounded corners, minimum width=3cm, minimum height=1cm,text centered, draw=black]
%\tikzstyle{question} = [rectangle, minimum width=3cm, minimum height=1cm, text centered, draw=black]
%\tikzstyle{decision} = [trapezium, trapezium left angle=70, trapezium right angle=110, minimum width=3cm, minimum height=1cm, text centered, draw=black]
%\tikzstyle{question} = [rectangle, rounded corners, minimum width=3cm, minimum height=1cm,text centered, draw=black]
%\tikzstyle{process} = [rectangle, minimum width=3cm, minimum height=1cm, text centered, draw=black]
%\tikzstyle{decision} = [trapezium, trapezium left angle=70, trapezium right angle=110, minimum width=3cm, minimum height=1cm, text centered, draw=black]


\outcome{Define a vector field.}
\outcome{Identify radial vector fields.}
\outcome{Identify gradient vector fields.}
\outcome{Find potential functions for gradient fields.}

\title[Dig-In:]{Vector fields}

\begin{document}
\begin{abstract}
  We introduce the idea of a vector at every point in space.
\end{abstract}
\maketitle

\section{Types of functions}

When we started on our journey exploring calculus, we investigated
functions $f:\R \to \R$.  Typically, we interpret these functions as
being curves in the $(x,y)$-plane:
\begin{image}
\begin{tikzpicture}
	\begin{axis}[
            xmin=-3.25,xmax=3.25,ymin=-1.5,ymax=1.5,
            axis lines=center,
            xtick={-6.28, -4.71, -3.14, -1.57, 0, 1.57, 3.142, 4.71, 6.28},
            xticklabels={$-2\pi$,$-3\pi/2$,$-\pi$, $-\pi/2$, $0$, $\pi/2$, $\pi$, $3\pi/2$, $2\pi$},
            ytick={-1,1},
            %ticks=none,
            width=3in,
            height=3in,
            unit vector ratio*=1 1 1,
            xlabel=$x$, ylabel=$y$,
            every axis y label/.style={at=(current axis.above origin),anchor=south},
            every axis x label/.style={at=(current axis.right of origin),anchor=west},
          ]
          \addplot [very thick, penColor, samples=100,smooth, domain=(-3.25:3.25)] {sin(deg(x))};
        \end{axis}
\end{tikzpicture}
%% \caption{The function $\sin(\theta)$ takes on all values between $-1$
%%   and $1$ exactly once on the interval $[-\pi/2,\pi/2]$. If we
%%   restrict $\sin(\theta)$ to this interval, then this restricted
%%   function has an inverse.}
%% \label{figure:sin-restricted}
%% \end{figure*}
\end{image}
We've also studied vector-valued functions $\vec{f}:\R \to \R^n$.  We
can interpret these functions as parametric curves in space:
\begin{image}
    \begin{tikzpicture}
      \begin{axis}%
        [tick label style={font=\scriptsize},axis on top,
	  axis lines=center,
	  view={155}{10},no markers,
	  ymin=-1.1,ymax=1.1,
	  xmin=-7,xmax=7,
	  zmin=-1.1, zmax=1.1,
	  every axis x label/.style={at={(axis cs:\pgfkeysvalueof{/pgfplots/xmax},0,0)},xshift=-3pt,yshift=-3pt},
	  xlabel={\scriptsize $x$},
	  every axis y label/.style={at={(axis cs:0,\pgfkeysvalueof{/pgfplots/ymax},0)},xshift=5pt,yshift=-2pt},
	  ylabel={\scriptsize $y$},
	  every axis z label/.style={at={(axis cs:0,0,\pgfkeysvalueof{/pgfplots/zmax})},xshift=0pt,yshift=4pt},
	  zlabel={\scriptsize $z$}
	]
        \draw[thick,->,penColor!50!white] (axis cs: 0,0,0)--(axis cs: {-4},{sin(deg(-4))},{cos(deg(-4))});
        \draw[thick,->,penColor!50!white] (axis cs: 0,0,0)--(axis cs: {-3},{sin(deg(-3))},{cos(deg(-3))});
        \draw[thick,->,penColor!50!white] (axis cs: 0,0,0)--(axis cs: {-2},{sin(deg(-2))},{cos(deg(-2))});
        \draw[thick,->,penColor!50!white] (axis cs: 0,0,0)--(axis cs: {-1},{sin(deg(-1))},{cos(deg(-1))});
        \draw[thick,->,penColor!50!white] (axis cs: 0,0,0)--(axis cs: {0},{sin(deg(0))},{cos(deg(0))});
        \draw[thick,->,penColor!50!white] (axis cs: 0,0,0)--(axis cs: {1},{sin(deg(1))},{cos(deg(1))});
        \draw[thick,->,penColor!50!white] (axis cs: 0,0,0)--(axis cs: {2},{sin(deg(2))},{cos(deg(2))});
        \draw[thick,->,penColor!50!white] (axis cs: 0,0,0)--(axis cs: {3},{sin(deg(3))},{cos(deg(3))});
        \draw[thick,->,penColor!50!white] (axis cs: 0,0,0)--(axis cs: {4},{sin(deg(4))},{cos(deg(4))});
        \addplot3+[very thick, penColor, smooth,samples=200,samples y=0,domain=-12:12] ({x},{sin(deg(x))},{cos(deg(x)});
      \end{axis}
    \end{tikzpicture}
    \end{image}
We've also studied functions of several variables $F:\R^n \to \R$.  We
can interpret these functions as surfaces in $\R^{n+1}$.  For example if $n=2$, then $F:\R^2\to\R$ plots a surface in $\R^3$:
\begin{image}
  \begin{tikzpicture}
    \begin{axis}%
      [width=175pt,tick label style={font=\scriptsize},axis on top,
	axis lines=center,
	view={155}{30},
	name=myplot,
	xtick=\empty,
	ytick=\empty,
	ztick=\empty,
	ymin=-1.2,ymax=1.2,
	xmin=-1.2,xmax=1.2,
	zmin=-1.2, zmax=1.2,
	every axis x label/.style={at={(axis cs:\pgfkeysvalueof{/pgfplots/xmax},0,0)},xshift=-3pt,yshift=-3pt},
	xlabel={\scriptsize $x$},
	every axis y label/.style={at={(axis cs:0,\pgfkeysvalueof{/pgfplots/ymax},0)},xshift=5pt,yshift=-2pt},
	ylabel={\scriptsize $y$},
	every axis z label/.style={at={(axis cs:0,0,\pgfkeysvalueof{/pgfplots/zmax})},xshift=0pt,yshift=4pt},
	zlabel={\scriptsize $z$},
        colormap/cool
      ]

      \addplot3[domain=-1:1,smooth,y domain=-1:1,mesh,samples=20,samples y=25,very thin,z buffer=sort] ({x},{y},{x^2-y^2});
    \end{axis}
  \end{tikzpicture}
\end{image}

Now we are ready for a new type of function.

\section{Vector fields}


Now we will study vector-valued functions of several variables:
\[
\vec{F}:\R^n\to \R^n
\]
We interpret these functions as \textit{vector fields}, meaning for
each point in the $(x,y)$-plane we have a vector.
\begin{image}
  \includegraphics{egField.png}
\end{image}
To some extent functions like this have been around us for a while,
for if
\[
G:\R^n\to\R
\]
then $\grad G$ is a vector-field.  Let's be explicit and write a definition.
\begin{definition}
  A \dfn{vector field} in $\R^n$ is a function
  \[
  \vec{F}: \R^n\to\R^n
  \]
  where for every point in the domain, we assign a vector to the range.
\end{definition}

\begin{question}
  Consider the following table describing a vector field $\vec{F}$:
  \begin{image}
    \begin{tikzpicture}[x=.75cm,y=.5cm]
      \draw (0,0) grid [step=1] (1,4);
      \draw (3,4) -- (3,0);
      \draw (5,4) -- (5,0);
      \draw (7,4) -- (7,0);
      \draw (9,4) -- (9,0);

      \draw (0,0) -- (9,0);
      \draw (0,1) -- (9,1);
      \draw (0,2) -- (9,2);
      \draw (0,3) -- (9,3);
      \draw (0,4) -- (9,4);

      \draw[ultra thick] (0,1)--(9,1);
      \draw[ultra thick] (1,4)--(1,0);

      \draw (0,0) -- (1,1);
      %\node at (.9,.9) [below left,inner sep=1pt] {\small$y$}; 
      %\node at (0.1,.1) [above right,inner sep=1pt] {\small$x$};
      \node at (.1,.9) [below right,inner sep=1pt] {\small$y$};
      \node at (0.9,.1) [above left,inner sep=1pt] {\small$x$};


      %% x-values
      \node at (2,.5) {$1$};
      \node at (4,.5) {$2$};
      \node at (6,.5) {$3$};
      \node at (8,.5) {$4$};

      %% y-values
      \node at (0.5,1.5) {$5$};
      \node at (0.5,2.5) {$6$};
      \node at (0.5,3.5) {$7$};


      %% vectors
      %% bottom row
      \node at (2,1.5) {$\veci+2\vecj$};
      \node at (4,1.5) {$-3\vecj$};
      \node at (6,1.5) {$2\vecj$};
      \node at (8,1.5) {$2\veci+4\vecj$};

      %% second row
      \node at (2,2.5) {$3\vecj$};
      \node at (4,2.5) {$\vec{0}$};
      \node at (6,2.5) {$2\veci-3\vecj$};
      \node at (8,2.5) {$\veci-2\vecj$};

      %% third row
      \node at (2,3.5) {$\veci-\vecj$};
      \node at (4,3.5) {$-2\vecj$};
      \node at (6,3.5) {$-2\veci-2\vecj$};
      \node at (8,3.5) {$\veci$};

    \end{tikzpicture}
  \end{image}
  What is $\vec{F}(1,7)$?
  \begin{prompt}
    \[
    \vec{F}(1,7) = \vector{\answer{1},\answer{-1}}
    \]
  \end{prompt}
  \begin{question}
    What is $\vec{F}(3,5)$?
    \begin{prompt}
      \[
      \vec{F}(3,5) = \vector{\answer{0},\answer{2}}
      \]
    \end{prompt}
    \begin{question}
      What is $\vec{F}(2,6)$?
      \begin{prompt}
        \[
        \vec{F}(2,6) = \vector{\answer{0},\answer{0}}
        \]
      \end{prompt}
    \end{question}
  \end{question}
\end{question}

\begin{question}
  Consider the following picture:
  \begin{image}
    \includegraphics{constField.png}
  \end{image}
  Which vector field is illustrated by this picture?
  \begin{multipleChoice}
    \choice{$\vec{F}(x,y)=\vector{x,y/2}$}
    \choice{$\vec{F}(x,y)=1/2$}
    \choice{$\vec{F}(x,y)=x+y/2$}
    \choice[correct]{$\vec{F}(x,y)=\vector{1,1/2}$}
  \end{multipleChoice}
  \begin{feedback}[correct]
    Note that with the first choice, the lengths of the vectors is
    changing, and that does not appear to be the case with our vector
    field.

    The second choice is not a vector field.

    The third choice is not a vector field.

    The fourth choice is a constant vector field, and is the correct answer.
  \end{feedback}
\end{question}



\section{Properties of vector fields}

As we will see in the chapters to come, there are two important
qualities of vector fields that we are usually on the look-out
for. The first is \textit{rotation} and the second is
\textit{expansion}. In the sections to come, we will make precise
what we mean by \textit{rotation} and \textit{expansion}. In this
section we simply seek to make you aware that these are the
fundamental properties of vector fields.

\subsection{Radial fields}

Very loosely speaking a \textit{radial field} is one where the vectors
are all pointing toward a spot, or away from a spot.  Let's see some
examples of radial vector fields.
\begin{example}
  Here we see $\vec{F}(x,y) =
  \vector{\frac{x}{\sqrt{x^2+y^2}},\frac{y}{\sqrt{x^2+y^2}}}$.
  \begin{image}
    \includegraphics{radField1.png}
  \end{image}
Those vectors are all pointing away from the central point!
\end{example}


\begin{example}
  Here we see $\vec{G}(x,y) =
  \vector{\frac{-x}{x^2+y^2},\frac{-y}{x^2+y^2}}$.
  \begin{image}
    \includegraphics{radField2.png}
  \end{image}
  Those vectors are all pointing toward the central point.
\end{example}

\begin{example}
  Here we see $\vec{H}(x,y,z) = \vector{x,y,z}$.
  \begin{image}
    \includegraphics{radField3.png}
  \end{image}
  This is a three-dimensional vector field where all the vectors are
  pointing away from the central point.
\end{example}

Each of the vector fields above is a \textit{radial} vector
field. Let's give an explicit definition.

\begin{definition}
  A \dfn{radial vector field} is a field of the form
  $\vec{F}:\R^n\to\R^n$ where
  \[
  \vec{F}(\vec{x}) = \frac{\pm\vec{x}}{|\vec{x}|^p}
  \]
  where $p$ is a real number.
\end{definition}

Fun fact: Newton's law of gravitation defines a radial vector field.


\begin{question}
  Is $\vec{F}(x,y,z) = \vector{x,y,z}$ a radial vector field?
  \begin{prompt}
    \begin{multipleChoice}
      \choice[correct]{yes}
      \choice{no}
    \end{multipleChoice}
    \begin{feedback}[correct]
      Absolutely! This vector field can be rewritten as:
      \[
      \vector{\frac{x}{(\sqrt{x^2+y^2+z^2})^p},\frac{y}{(\sqrt{x^2+y^2+z^2})^p},\frac{z}{(\sqrt{x^2+y^2+z^2})^p}}
      \]
      where $p=\answer{0}$.
    \end{feedback}
  \end{prompt}
\end{question}

Some fields look like they are expanding and are. Other fields look
like the are expanding but they aren't. In the sections to come, we're
going to use calculus to precisely define what we mean by a field
``expanding.'' This property will be called \textit{divergence}.




\subsection{Rotational fields}

Vector fields can easily exhibit what looks like ``rotation'' to the
human eye. Let's show you a few examples.
\begin{example}
  Here we see $\vec{F}(x,y) = \vector{-y,x}$.
  \begin{image}
    \includegraphics{rotField.png}
  \end{image}
  This vector field looks like it has counterclockwise rotation.
\end{example}

\begin{example}
  Here we see
  \[
  \vec{F}(x,y) = \vector{\frac{-y}{x^2+y^2},\frac{x}{x^2+y^2}}:
  \]
  \begin{image}
    \includegraphics{rotFieldNot.png}
  \end{image}
  This vector field looks like it has clockwise rotation.
\end{example}

At this point, we're going to give some ``spoilers.'' It turns out
that from a local perspective, meaning looking at points very very
close to each other, only the first example exhibits ``rotation.''
While the second example looks like it is ``rotating,'' as we will
see, it does not exhibit ``local rotation.''  Moreover, in future
sections we will see that rotation (even local rotation) in
three-dimensional space must always happen around some ``axis'' like
this:
  \begin{image}
    \includegraphics{rotField3.png}
  \end{image}
In the sections to come, we will use calculus to precisely explain
what we mean by ``local rotation.'' This property will be called
\textit{curl}.





\section{Gradient fields}

In this final section, we will talk about fields that arise as the
gradient of some differentiable function.  As we will see in future
sections, these are some of the nicest vector fields to work with
mathematically.


\begin{definition}
  Consider any differentiable function $F:\R^n\to \R$.
  A \dfn{gradient field} is a vector field $\vec{G}:\R^n\to \R^n$ where
  \[
  \vec{G} = \grad F.
  \]
  Note, since we are assuming $F$ is differentiable, we are also
  assuming that $\vec{G}$ is defined for all points in $\R^n$.
\end{definition}

Let's take a look at a gradient field.


\begin{example}
  Consider $F(x,y) = \frac{\sin(3x)+\sin(2y)}{1+x^2+y^2}$. A plot of
  this function looks like this:
  \begin{image}
    \includegraphics{surf1.png}
  \end{image}
  The gradient field of $F$ looks like this:
  \begin{image}
    \includegraphics{gradField1.png}
  \end{image}
  Note we can see the vector pointing in the initial direction of
  greatest increase. Let's see a plot of both together:
  \begin{image}
    \includegraphics{gradSurf1.png}
  \end{image}
\end{example}



\begin{question}
  Remind me, what direction do gradient vectors point?
  \begin{prompt}
    \begin{multipleChoice}
      \choice{Gradient vectors point to the maximum.}
      \choice{Gradient vectors point up.}
      \choice[correct]{Gradient vectors point in the initial direction of greatest increase.}
    \end{multipleChoice}
  \end{prompt}
\end{question}

\begin{example}
  Now consider $F(x,y) = \frac{1}{\sqrt{x^2+y^2}}$. A plot of this
  function looks like this:
  \begin{image}
    \includegraphics{surf2.png}
  \end{image}
  The gradient field of $F$
  \[
  \grad F(x,y) = \vector{\frac{-x}{(x^2+y^2)^{3/2}},\frac{-y}{(x^2+y^2)^{3/2}}}
  \]
  looks like this:
  \begin{image}
    \includegraphics{gradField2.png}
  \end{image}
  Note we can see the vector pointing in the initial direction of
  greatest increase. Let's see a plot of both together:
  \begin{image}
    \includegraphics{gradSurf2.png}
  \end{image}
\end{example}




\section{The shape of things to come}

Now we present the beginning of a big idea. By the end of this course,
we hope to give you a glimpse of ``what's out there.'' For this we're
going to need some notation. Think of $A$ and $B$ as sets of numbers,
like $A=\R$ or $A=\R^n$ or $B=\R$ or $B=\R^n$.

\begin{itemize}
\item $C(A,B)$ is the set of continuous functions from $A$ to $B$.
\item $C^1(A,B)$ is the set of  differentiable functions from $A$ to $B$ whose
  first-derivative is continuous.
\item $C^2(A,B)$ is the set of  differentiable functions from $A$ to $B$ whose
  first and second derivatives are continuous.
\item $C^n(A,B)$ is the set of differentiable function from $A$ to $B$
  where the first $n$th derivatives are continuous.
\item $C^\infty(A,B)$ is the set of differentiable functions from $A$
  to $B$ where \textbf{all} of the derivatives are continuous.
\end{itemize}
Here is a deep idea:
\begin{quote}
  \textbf{The gradient turns functions of several variables into vector fields.}
\end{quote}
We can write this with our new notation as:
\[
C^\infty(\R^n,\R)\overset{\grad}{\to}C^\infty(\R^n,\R^n)
\]


\subsection{The Clairaut gradient test}


Now we give a method to determine if a field is a gradient field.


\begin{theorem}[Clairaut]
  A vector field $\vec{G}(x,y) = \vector{M(x,y),N(x,y)}$, where $M$ and
  $N$ have continuous partial derivatives, is a gradient field if and
  only if
  \[
  \pp[N]{x} -\pp[M]{y} = 0
  \]
  for all $x$ and $y$.
  \begin{explanation}
    Let's take a second and think about the gradient as a function on
    functions. Let $C^\infty(A,B)$ be the set of all function from $A$
    to $B$ whose $i$th-derivatives are continuous for all values of
    $i$. The gradient takes functions of several variables and maps
    them to vector fields:
\begin{image}
  \begin{tikzpicture}
    \node at (0,0) {$\underbrace{C^\infty(\R^2,\R)}_{\text{functions of several variables}}$};
    \draw[->] (1,.25) -- (3,.25);
    \node at (2,.6) {$\grad$};
    \node at (4,0) {$\underbrace{C^\infty(\R^2,\R^2)}_{\text{vector fields}}$};
  \end{tikzpicture}
\end{image}
So $\vec{G}(x,y) = \vector{M(x,y),N(x,y)}$ if and only if there is
some function $F:\R^2\to\R$ where
\[
\pp[F]{x} = M \quad \text{and}\quad \pp[F]{y} = N
\]
but if all the partial derivatives are continuous, then:
\[
\pp[M]{y} = \frac{\partial^2 F}{\partial y\partial x} =  \frac{\partial^2 F}{\partial x\partial y} = \pp[N]{x}
\]
This is true if and only if $\pp[N]{x} -\pp[M]{y} = 0$.
  \end{explanation}
\end{theorem}


\begin{definition}
  If $\vec{G} = \grad F$, then $F$ is called a \dfn{potential
    function} for $\vec{G}$.
\end{definition}


\begin{example}
  Is $\vec{G} = \vector{2x+3y,2+3x+2y}$ a gradient field? If so find a
  potential function.
  \begin{explanation}
    To start, let's do Clairaut's gradient test:
    \begin{align*}
      M(x,y) &= \answer[given]{2x+3y}\\
      N(x,y) &= \answer[given]{2+3x+2y}
    \end{align*}
    And so
    \begin{align*}
      \pp[N]{x} &= \answer[given]{3}
      \pp[M]{y} &= \answer[given]{3}
    \end{align*}
    And so we see $\pp[N]{x} -\pp[M]{y} = 0$, and thus $\vec{G}$ is a
    gradient field. Now let's try to find a potential function. To do
    this, we'll antidifferentiate---in essence we want to ``undo'' the
    gradient. Write with me:
    \begin{align*}
      \int M(x,y) \d x &= \int \left(\answer[given]{2x+3y}\right) \d x \\
      &= \answer[given]{x^2+3xy} + c(y)
    \end{align*}
    where $C(y)$ is a function of $y$. In a similar way:
    \begin{align*}
      \int N(x,y) \d y &= \int \left(\answer[given]{2+3x+2y}\right) \d y \\
      &= \answer[given]{2y + 3xy+ y^2} + k(x)
    \end{align*}
    We need
    \[
    x^2+3xy + c(y) = 2y + 3xy+ y^2 + k(x)
    \]
    to make this happen, we set $c(y) = 2y+y^2$ and $k(x) = x^2$. From
    this we find our potential function is $F(x,y) = x^2+3xy+ y^2 + 2y$.
  \end{explanation}
\end{example}

Now try your hand at these questions:

\begin{question}
  Is $\vec{G} = \vector{2x+y^2,2y+x^2}$ a gradient field? If so find a
  potential function.
  \begin{prompt}
  \begin{multipleChoice}
    \choice{yes}
    \choice[correct]{no}
  \end{multipleChoice}
  \end{prompt}
\end{question}


\begin{question}
  Is $\vec{G} = \vector{x^3,-y^4}$ a gradient field? If so find a
  potential function.
  \begin{prompt}
  \begin{multipleChoice}
    \choice[correct]{yes}
    \choice{no}
  \end{multipleChoice}
  \begin{question}
    Find a potential function $F$ such that $F(\vec{0}) = 0$.
    \begin{prompt}
      \[
      F(x,y) = \answer{x^4/4-y^5/5}.
      \]
    \end{prompt}
  \end{question}
  \end{prompt}
\end{question}

\begin{question}
  Is $\vec{G} = \vector{y\cos(x),\sin(x)}$ a gradient field? If so find a
  potential function.
  \begin{prompt}
  \begin{multipleChoice}
    \choice[correct]{yes}
    \choice{no}
  \end{multipleChoice}
  \begin{question}
    Find a potential function $F$ such that $F(\vec{0}) = 0$.
    \begin{prompt}
      \[
      F(x,y) = \answer{y\sin(x)}.
      \]
    \end{prompt}
  \end{question}
  \end{prompt}
\end{question}

\begin{question}
  Is $\vec{G} = \vector{\frac{-y}{x^2+y^2},\frac{x}{x^2+y^2}}$ a
  gradient field? If so find a potential function.
  \begin{prompt}
  \begin{multipleChoice}
    \choice{yes}
    \choice[correct]{no}
  \end{multipleChoice}
  \begin{feedback}
    What happens at $\vec{0}$?
  \end{feedback}
  \end{prompt}
\end{question}


\end{document}
